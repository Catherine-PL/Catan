\documentclass[a4paper, 11pt]{article}
\usepackage[polish]{babel}
\usepackage[MeX]{polski}
\usepackage[utf8]{inputenc}
\usepackage[T1]{fontenc}
%\usepackage{times}
\usepackage{graphicx,wrapfig}
%\usepackage{anysize}
%\usepackage{tikz}
%\usetikzlibrary{calc,through,backgrounds,positioning}
\usepackage{anysize}
\usepackage{float}
%\usepackage{stmaryrd}
%\usepackage{amssymb}
%\usepackage{amsthm}
%\marginsize{3cm}{3cm}{3cm}{3cm}
%\usepackage{amsmath}
%\usepackage{color}
%\usepackage{listings}
%\usepackage{enumerate}
%\lstloadlanguages{Ada,C++}


\begin{document}	
	% \noindent -  w tym akapicie nie bedzie wciecia
	% \ indent - to jest aut., ale powoduje ze jest wciecie
	% \begin{flushleft}, flushright, center - wyrownianie akapitu
	% \textbf{pogrubiany tekst}
	% \textit{kursywa} 
	% 					STRONY 
	%  http://www.codecogs.com/latex/eqneditor.php 
	%  http://www.matematyka.pl/latex.htm
	% 
	\begin{titlepage}
	
	
		
		\newcommand{\HRule}{\rule{\linewidth}{0.5mm}} % Defines a new command for the horizontal lines, change thickness here
		
		\center % Center everything on the page
		
		%----------------------------------------------------------------------------------------
		%	HEADING SECTIONS
		%----------------------------------------------------------------------------------------
		
		\textsc{\LARGE Akademia Górniczo-Hutnicza im. Stanisława Staszica}\\[1.5cm] % Name of your university/college
		\textsc{\Large Kraków}\\[0.5cm] % Major heading such as course name
		\textsc{\large }\\[0.5cm] % Minor heading such as course title
		
		%----------------------------------------------------------------------------------------
		%	TITLE SECTION
		%----------------------------------------------------------------------------------------
		
		\HRule \\[0.4cm]
		{\fontsize{38}{50}\selectfont Osadnicy z Catan - Gra sieciowa}
	%	{ \Huge \bfseries} Osadnicy z Catan - Gra sieciowa\\[0.3cm] % Title of your document
		\HRule \\[1.5cm]
		
		%----------------------------------------------------------------------------------------
		%	AUTHOR SECTION
		%----------------------------------------------------------------------------------------
		
		% If you don't want a supervisor, uncomment the two lines below and remove the section above
		\Large \emph{Autorzy:}\\
		Marcin \textsc{Jędrzejczyk}\\ % Your name
		Sebastian \textsc{Katszer}\\ % Your name
		Katarzyna \textsc{Kosiak} \\
		Paweł \textsc{Ogorzały}\\[3cm]\ % Your name

		
		%----------------------------------------------------------------------------------------
		%	DATE SECTION
		%----------------------------------------------------------------------------------------
		
		{\large \today}\\[3cm] % Date, change the \today to a set date if you want to be precise
		
		%----------------------------------------------------------------------------------------
		%	LOGO SECTION
		%----------------------------------------------------------------------------------------
		
		%\includegraphics{Logo}\\[1cm] % Include a department/university logo - this will require the graphicx package
		
		%----------------------------------------------------------------------------------------
		
		\vfill % Fill the rest of the page with whitespace
		
	\end{titlepage}
	
	%SPIS TRESI
	%
	%
	%
	%
	%
	%
	%
	%
	%
	
	\tableofcontents
	\vfill
	\newpage	%\pagebreak
	
	%SEKCJE
	%opis zagadnienia, temat, problem, dlaczego chcemy to rozwiązacyać metodą ewolucyjnę
	%jakie są metody rozwiącania problemu, przegląd literatury,
	%proponowane rozwiązania (spójność)
	%czym się inspierowałyśmy
	%
	%
	%
	
	%\setlength{\parskip}{1ex plus 0.5ex minus 0.2ex}
	
	\section{Wstęp}
	\indent
	
	Niniejszy dokument stanowi dokumentację projektu ``Osadnicy z Catanu - Gra sieciowa'' z przedmiotu Wprowadzenie do Wzorców Projektowych.
	Nasz projekt zrealizowaliśmy
	\subsection{Dlaczego Catan?}
	\indent 
	
	Nasza grupa projektowa składa się z osób zafascynowanych światem zarówno gier planszowych jak i komputerowych. Projekt ten umożliwia nam połączenie naszych zainteresowań. Wybór padł na grę ``Osadnicy z Catanu'' z bardzo prostego powodu: dzięki prostocie i przejrzystości zasad stanowi ona idealny start dla osób, które chcą rozpocząć swoją przygodę z grami planszowymi. \\ 
	
	Dzięki wprowadzeniu rozgrywki sieciowej możliwe jest rozegranie partii z przyjaciółmi z całego świata bez dodatkowych wydatków, a także rozpowszechnienie tej gry i ułatwienie wejścia w świat gier planszowych. 
	\subsection{Opis gry}
	\indent
	
Osadnicy z Catanu (Settlers of Catan) to jedna z najpopularniejszych rodzinno-ekonomicznych gier planszowych na świecie. 
%do cytatu
Gracze są osadnikami na niedawno odkrytej wyspie Catan. Każdy z nich przewodzi świeżo założonej kolonii i rozbudowuje ją stawiając na dostępnych obszarach nowe drogi i miasta. Każda kolonia zbiera dostępne dobra naturalne, które są niezbędne do rozbudowy osiedli.
 
Gracz musi rozważnie stawiać nowe osiedla i drogi, aby zapewnić sobie dostateczny, ale zrównoważony dopływ zasobów, a jeśli ma ich nadmiar - prowadzić handel z innymi graczami sprzedając im owce, drewno, cegły, zboże lub żelazo a pozyskując od nich te zasoby, których ciągle mu brakuje.
 
Pierwszy z graczy, który uzyska 10 punktów z wybudowanych przez siebie dróg, osiedli i specjalnych kart - wygrywa.
	
%skrócone zasady	

%	automaty gdzies!!!!!!\\
	
	
	
	
	
	\section{Użyte wzorce projektowe}
	\subsection{Singleton}
	kostka \\
	Singleton Pattern says that just"define a class that has only one instance and provides a global point of access to it".

In other words, a class must ensure that only single instance should be created and single object can be used by all other classes.

	\subsection{Builder}
	kafelki,plansza\\
	\subsection{Abstract Factory}
	do buildera???,generowanie kart itp\\
	\subsection{Observer}
	komunikacja\\
	\subsection{State}
	złodziej,karty specjalne, drogi\\
	\subsection{Command}
		fghfghfgh\\gfhfghfgh
	\section{Użyte biblioteki zewnętrzne}
	\subsection{Biblioteka graficzna}
	libgtx\\
	 https://libgdx.badlogicgames.com/
	\subsection{Biblioteka sieciowa}
	jxta\\
	https://jxta.kenai.com/
	\section{Architektura}
	\indent
	
	Poniżej prezentujemy opis wszystkich warstw, na które podzieliliśmy naszą aplikację.
	%screen z głównym uml
	\begin{figure}[H]%[!htb]
		\includegraphics[scale=0.5]{uml/main.jpg}\caption{Podział na warstwy}
	\end{figure}

	\subsection{Warstwa prezentacji}	
	\begin{figure}[H]%[!htb]
		\includegraphics[scale=0.5]{uml/main.jpg}
	\end{figure}
	\subsection{Warstwa interakcji}
	\begin{figure}[H]%[!htb]
		\includegraphics[scale=0.5]{uml/main.jpg}
	\end{figure}	
	\subsection{Warstwa logiki gry}	
	\begin{figure}[H]%[!htb]
		\includegraphics[scale=0.5]{uml/main.jpg}
	\end{figure}
	\subsection{Warstwa baz danych}	
	\begin{figure}[H]%[!htb]
		\includegraphics[scale=0.5]{uml/main.jpg}
	\end{figure}
	\subsection{Warstwa komunikacji sieciowej}
	\begin{figure}[H]%[!htb]
		\includegraphics[scale=0.5]{uml/main.jpg}
	\end{figure}	
	
	
	
	\section{Podsumowanie}
	

	\section{Testy}
	

	\section{Wnioski}
	

	\section{Literatura}
	
\textbf{Asensio MI, Ferragut L., Simon J.:} Modelling of convective phenomena in forest fire. Rev Real Academia de Ciencias, 2002, 96:299–313\\


\end{document}
