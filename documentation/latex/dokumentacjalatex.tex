\documentclass[a4paper, 11pt]{article}
\usepackage[polish]{babel}
\usepackage[MeX]{polski}
\usepackage[utf8]{inputenc}
\usepackage[T1]{fontenc}
%\usepackage{times}
\usepackage{graphicx,wrapfig}
%\usepackage{anysize}
%\usepackage{tikz}
%\usetikzlibrary{calc,through,backgrounds,positioning}
\usepackage{anysize}
\usepackage{float}
%\usepackage{stmaryrd}
%\usepackage{amssymb}
%\usepackage{amsthm}
%\marginsize{3cm}{3cm}{3cm}{3cm}
%\usepackage{amsmath}
%\usepackage{color}
%\usepackage{listings}
%\usepackage{enumerate}
%\lstloadlanguages{Ada,C++}


\begin{document}	
	% \noindent -  w tym akapicie nie bedzie wciecia
	% \ indent - to jest aut., ale powoduje ze jest wciecie
	% \begin{flushleft}, flushright, center - wyrownianie akapitu
	% \textbf{pogrubiany tekst}
	% \textit{kursywa} 
	% 					STRONY 
	%  http://www.codecogs.com/latex/eqneditor.php 
	%  http://www.matematyka.pl/latex.htm
	% 
	\begin{titlepage}
	
	
		
		\newcommand{\HRule}{\rule{\linewidth}{0.5mm}} % Defines a new command for the horizontal lines, change thickness here
		
		\center % Center everything on the page
		
		%----------------------------------------------------------------------------------------
		%	HEADING SECTIONS
		%----------------------------------------------------------------------------------------
		
		\textsc{\LARGE Akademia Górniczo-Hutnicza im. Stanisława Staszica}\\[1.5cm] % Name of your university/college
		\textsc{\Large Kraków}\\[0.5cm] % Major heading such as course name
		\textsc{\large }\\[0.5cm] % Minor heading such as course title
		
		%----------------------------------------------------------------------------------------
		%	TITLE SECTION
		%----------------------------------------------------------------------------------------
		
		\HRule \\[0.4cm]
		{\fontsize{38}{50}\selectfont Osadnicy z Catan - Gra sieciowa}
	%	{ \Huge \bfseries} Osadnicy z Catan - Gra sieciowa\\[0.3cm] % Title of your document
		\HRule \\[1.5cm]
		
		%----------------------------------------------------------------------------------------
		%	AUTHOR SECTION
		%----------------------------------------------------------------------------------------
		
		% If you don't want a supervisor, uncomment the two lines below and remove the section above
		\Large \emph{Autorzy:}\\
		Marcin \textsc{Jędrzejczyk}\\ % Your name
		Sebastian \textsc{Katszer}\\ % Your name
		Katarzyna \textsc{Kosiak} \\
		Paweł \textsc{Ogorzały}\\[3cm]\ % Your name

		
		%----------------------------------------------------------------------------------------
		%	DATE SECTION
		%----------------------------------------------------------------------------------------
		
		{\large \today}\\[3cm] % Date, change the \today to a set date if you want to be precise
		
		%----------------------------------------------------------------------------------------
		%	LOGO SECTION
		%----------------------------------------------------------------------------------------
		
		%\includegraphics{Logo}\\[1cm] % Include a department/university logo - this will require the graphicx package
		
		%----------------------------------------------------------------------------------------
		
		\vfill % Fill the rest of the page with whitespace
		
	\end{titlepage}
	
	%SPIS TRESI
	%
	%
	%
	%
	%
	%
	%
	%
	%
	
	\tableofcontents
	\vfill
	\newpage	%\pagebreak
	
	%SEKCJE
	%opis zagadnienia, temat, problem, dlaczego chcemy to rozwiązacyać metodą ewolucyjnę
	%jakie są metody rozwiącania problemu, przegląd literatury,
	%proponowane rozwiązania (spójność)
	%czym się inspierowałyśmy
	%
	%
	%
	
	%\setlength{\parskip}{1ex plus 0.5ex minus 0.2ex}
	
	\section{Wstęp}
	\indent
	
	Niniejszy dokument stanowi dokumentację projektu ``Osadnicy z Catanu - Gra sieciowa'' z przedmiotu Wprowadzenie do Wzorców Projektowych.
	Nasz projekt zrealizowaliśmy
	\subsection{Dlaczego Catan?}
	\indent 
	
	Nasza grupa projektowa składa się z osób zafascynowanych światem zarówno gier planszowych jak i komputerowych. Projekt ten umożliwia nam połączenie naszych zainteresowań. Wybór padł na grę ``Osadnicy z Catanu'' z bardzo prostego powodu: dzięki prostocie i przejrzystości zasad stanowi ona idealny start dla osób, które chcą rozpocząć swoją przygodę z grami planszowymi. \\ 
	
	Dzięki wprowadzeniu rozgrywki sieciowej możliwe jest rozegranie partii z przyjaciółmi z całego świata bez dodatkowych wydatków, a także rozpowszechnienie tej gry i ułatwienie wejścia w świat gier planszowych. 
	\subsection{Opis gry}
	\indent
	
Osadnicy z Catanu (Settlers of Catan) to jedna z najpopularniejszych rodzinno-ekonomicznych gier planszowych na świecie. 
%do cytatu
Gracze są osadnikami na niedawno odkrytej wyspie Catan. Każdy z nich przewodzi świeżo założonej kolonii i rozbudowuje ją stawiając na dostępnych obszarach nowe drogi i miasta. Każda kolonia zbiera dostępne dobra naturalne, które są niezbędne do rozbudowy osiedli.
 
Gracz musi rozważnie stawiać nowe osiedla i drogi, aby zapewnić sobie dostateczny, ale zrównoważony dopływ zasobów, a jeśli ma ich nadmiar - prowadzić handel z innymi graczami sprzedając im owce, drewno, cegły, zboże lub żelazo a pozyskując od nich te zasoby, których ciągle mu brakuje.
 
Pierwszy z graczy, który uzyska 10 punktów z wybudowanych przez siebie dróg, osiedli i specjalnych kart - wygrywa.
	
%skrócone zasady	
\section{Skrócone zasady gry}
\begin{itemize}
\item Wygrywa gracz, który jako pierwszy zdobędzie 10 punktów zwycięstwa,
\item Gracz może handlować z bankiem lub z innymi graczami, ale może  zainicjować handel tylko podczas swojej tury,
\item Na wierzchołkach sąsiadujących z miastami lub osadami nie można stawiać nowych budynków,
\item Do nowej osady musi prowadzić droga. Wyjątek stanowią 2 pierwsze osady gracza,
\item Dwie pierwsze osady gracza są darmowe. Dwie pierwsze drogi gracza także są darmowe,
\item Na pustym polu, na którym można budować, można postawić osadę lub miasto (z wliczoną ceną osady),
\item Miasta mogą nadbudowywać osady,
\item Nowa droga musi się zaczynać w mieście, osadzie lub być kontynuacją innej drogi, której gracz jest właścicielem,
\item Rzut kością decyduje o tym, które pola przyniosą zasoby w danej turze. Jedna osada produkuje 1 zasób, jedno miasto produkuje 2 zasoby,
\item Gracz, który ma więcej niż 5 żołnierzy dostaje kartę największej armii, kolejni gracze muszą mieć o 1 żołnierzy więcej, by odebrać tą kartę właścicielowi,
\item Gracz, którego nieprzerwana droga jest dłuższa niż 5 dostaje kartę najdłuższej drogi handlowej, kolejni gracze muszą mieć o 1 dłuższą nieprzerwaną drogę, aby odebrać tą kartę właścicielowi.
\end{itemize}

Co daje punkty zwycięstwa:
\begin{itemize}
\item Osada - 1 punkt zwycięstwa,
\item Miasto - 2 punkty zwycięstwa,
\item Karta punkty zwycięstwa - 1 punkt zwycięstwa,
\item Karta największej armii - 2 punkty zwycięstwa,
\item Karta najdłuższej drogi handlowej - 2 punkty zwycięstwa.
\end{itemize}

Wyjaśnienie kart rozwoju:
\begin{itemize}
\item Karta żołnierza-zwiększa liczebność armii gracza o 1, pozwala przesunąć złodzieja,
\item Karta monopolu- pozwala wybrać zmonopolizowany surowiec, wszyscy gracze muszą oddać cały stan tego surowca zagrywającemu tę kartę,
\item Karta urodzaju- pozwala wybrać 2 darmowe surowce,
\item Karta punktu zwycięstwa- zagrywający dostaje 1 punkt zwycięstwa,
\item Karta budowy dróg- zagrywający dostaje 2 darmowe drogi do wybudowania.
\end{itemize}
%kto da radę to napisać bardziej po polsku
Koszty:
\begin{itemize}
\item Osada-  1 drewna, 1 cegły, 1 wełny, 1 zboża,
\item Miasto-  3 rudy i 2 zboża lub 1 drewna, 1 cegły, 1 wełny, 3 zboża i 3 rudy,
\item Karta rozwoju-  1 rudy, 1 wełny, 1 zboża,
\item Droga-  1 drewno, 1 cegły,
\end{itemize}


	
	
	\section{Użyte wzorce projektowe}
	\subsection{Singleton}
\indent

Singleton jest kreacyjnym wzorcem projektowym, którego zadaniem jest ograniczenie możliwości tworzenia obiektów danej klasy do jednej instancji, oraz zapewnienie globalnego dostępu do stworzonego obiektu.

Singleton implementuje się poprzez stworzenie klasy, która posiada statyczną metodę, która sprawdza czy istnieje instancja danej klasy, w razie potrzeby tworząc ją. Następnie instancja zwracana jest przez referencję. Aby uniemożliwić tworzenie dodatkowych instancji, konstruktor klasy deklaruje się jako prywatny lub chroniony.

Wzorzec ten ma u nas zastosowanie w przypadku klas odpowiadających za kostkę, planszę. Zapewnia on globalny dostęp oraz ogranicza możliwość tworzenia większej liczby instancji. 

	\subsection{Builder}
	\indent
	
Builder jest wzorcem konstrukcyjnym, który ma za zadanie oddzielenie tworzenia obiektów od ich reprezentacji. 
Proces tworzenia obiektu podzielony jest na mniejsze etapy, a każdy z nich może być implementowany na wiele różnych sposobów. Umożliwia to tworzenie różnych reprezentacji obiektów w tym samym procesie konstrukcyjnym.
Konstruowanie obiektu następuje poprzez wcześniejsze stworzenie jego fragmentów.


Na wzorzec składa się:
\begin{itemize}

\item Builder - dostarcza interfejs do tworzenia obiektów nazywanych produktami,
\item ConcreteBuilder - tworzy konkretne reprezentacje produktów przy pomocy zaimplementowanego interfejsu Builder,
\item Director - zleca konstrukcję produktów poprzez obiekt Builder.
\end{itemize}

W naszym projekcie wzorzec ten znajduje zastosowanie przy tworzeniu planszy, która składa się z kafli.


	\subsection{Abstract Factory}
\indent

Kreacyjny wzorzec projektowy, który pozwala tworzyć całe rodziny produktów. Dostarcza on interfejs do tworzenia różnych obiektów jednego typu bez specyfikowania ich konkretnych klas.

%Wzorzec ten wykorzysytwany będzie przy konstruowaniu kafli oraz generowaniu kart.


	\subsection{Observer}
	Wzorzec ''Observer'' jest używany jeśli występuje relacja jeden do wielu pomiędzy obiektami. Modyfikacja jednego obiektu powoduje, że zależne obiekty są powiadamiane automatycznie. Wzorzec ten podchodzi pod kategorię wzorców czynnościowych.\\
	%Observer pattern is used when there is one-to-many relationship between objects such as if one object is modified, its depenedent objects are to be notified automatically. Observer pattern falls under behavioral pattern category.\\
%	"That’s what the observer pattern is for. It lets one piece of code announce that something interesting happened without actually caring who receives the notification."\\
%	komunikacja\\
Wzorzec ten pomoże nam w komunikacji między graczami. Będziemy wysyłać informacje o tym co się zmieniło na planszy. A OBSERVER każdego gracza będzie je wyłapywał i egzekwował.

\subsection{Decorator}
Wzorzec dekoratora polega na opakowaniu oryginalnej klasy w nową klasę "dekorującą". Zwykle przekazuje się oryginalny obiekt jako parametr konstruktora dekoratora, metody dekoratora wywołują metody oryginalnego obiektu i dodatkowo implementują nową funkcję.\\

\subsection{Prototype}
Konstrukcyjny wzorzec projektowy, którego celem jest umożliwienie tworzenia obiektów danej klasy bądź klas z wykorzystaniem już istniejącego obiektu, zwanego prototypem. Głównym celem tego wzorca jest uniezależnienie systemu od sposobu w jaki tworzone są w nim produkty.\\

\subsection{Facade}
Wzorzec projektowy należący do grupy wzorców strukturalnych. Służy do ujednolicenia dostępu do złożonego systemu poprzez wystawienie uproszczonego, uporządkowanego interfejsu programistycznego, który ułatwia jego użycie.\\
	%\subsection{State}
%	\indent
%We wzorcy ''State'' zachowanie klasy zmienia się w zależności od jej stanu. Ten typ wzorca wchodzi pod wzorce czynnościowe.\\
%	In State pattern a class behavior changes based on its state. This type of design pattern comes under behavior pattern.
%Tworzymy obiekty, które reprezentują różne stany i konteksty obiektu, którego zachowanie zmienia się tak jak jego stan.
%%In State pattern, we create objects which represent various states and a context object whose behavior varies as its state object changes.\\
%W zależności od tego jaką kartę zagramy stan planszy może się zmienić. Możemy wybudować drogi, wejść w interakcję z innymi graczami, czy zablokować dochód z danego pola. Dlatego potrzebujemy wzorca odpowiedzialnego za zajmowanie się stanem klasy.
%złodziej,karty specjalne, drogi\\
	%\subsection{Command}
	
	%Command pattern is a data driven design pattern and falls under behavioral pattern category. A request is wrapped under an object as command and passed to invoker object. Invoker object looks for the appropriate object which can handle this command and passes the command to the corresponding object which executes the command.
%Wzorzec ''Command'' opakowuje żądania w obiekty jako rozkaz i przesyła do obiektu wzywającego (z ang. invoker). Obiekt wzywający szuka obiektu właściwego, który może zająć się tym rozkazem i   przesyła rozkaz do odpowiedniego obiektu który wykonuje rozkaz.\\  
	%Wzorzec ten należy do wzorców czynnościowych, czyli opisujących sposób przepływu danych w złożonych aplikacjach.%to z Polskiej strony mam
	%??? Skorzystamy z niego, by móc zagrywać karty rozwoju, które są swoistymi rozkazami. Wzorzec ten będzie musiał u nas współpracować z kodem napisanym pod wzorzec ''State''. %??? co wy na to
%	TO DO gdzie z tego skorzystamy
	%Commands are an object-oriented replacement for callbacks.\\
	
	
	\section{Użyte biblioteki zewnętrzne}
	\subsection{Biblioteka graficzna}
	LibGDX to wieloplatformowe narzędzie do tworzenia gier i wizualizacji.\\
	 https://libgdx.badlogicgames.com/
	\subsection{Biblioteka sieciowa}
	Jxta jest to zbiór uogólnionych i otwartych protokołów pozwalającym na komunikację w stylu peer-to-peer. \\
	https://jxta.kenai.com/
	\section{Architektura}
	\indent
	
	Poniżej prezentujemy opis wszystkich warstw, na które podzieliliśmy naszą aplikację.
	%screen z głównym uml
	\begin{figure}[H]%[!htb]
		\includegraphics[scale=0.5]{uml/main.jpg}\caption{Podział na warstwy}
	\end{figure}

%	\subsection{Warstwa prezentacji}	
%	\begin{figure}[H]%[!htb]
%		\includegraphics[scale=0.5]{uml/main.jpg}
%	\end{figure}
%	\subsection{Warstwa interakcji}
%	\begin{figure}[H]%[!htb]
%		\includegraphics[scale=0.5]{uml/main.jpg}
%	\end{figure}	
%	\subsection{Warstwa logiki gry}	
%	\begin{figure}[H]%[!htb]
%		\includegraphics[scale=0.5]{uml/main.jpg}
%	\end{figure}
%	\subsection{Warstwa baz danych}	
%	\begin{figure}[H]%[!htb]
%		\includegraphics[scale=0.5]{uml/main.jpg}
%	\end{figure}
%	\subsection{Warstwa komunikacji sieciowej}
%	\begin{figure}[H]%[!htb]
%		\includegraphics[scale=0.5]{uml/main.jpg}
%	\end{figure}	
	
	
	
	\section{Podsumowanie}
	Wspólna praca pozwoliła nam poprawić zdolność do pracy w zespole i pozwoliła przekonać się jak zmiany w kodzie jednej osoby wpływają na działanie kodu drugiej. Podział pracy na kilka warstw pozwolił na nam działać trochę niezależnie i równolegle, jednakże skutkowało to większym nakładem pracy nad połączeniem poszczególnych części. Przekonaliśmy się jak  wzorce projektowe ułatwiają życie i czytelność kody. Ostatecznie udało się nam wyprodukować działająca grę, co cieszy nas niezmiernie.
	

	\section{Testy}
	Dzięki stałemu testowaniu, byliśmy w stanie wyłapywać różne błędy logiki gry na bieżąco. Niektóre wymagały zmian podejścia do danych problemów i innych metod nimi zarządzających. 

	\section{Wnioski}
	\begin{itemize}
	\item Po zdefiniowaniu problemu należy sprawdzić, czy jakiś wzorzec nie pomoże w jego rozwiązaniu,
	\item Nie stosować wzorców na siłę, wzorzec nie musi zawsze pomóc,
	\item Uprzednie zaprojektowanie (np. w UMLu) projektu bardzo by pomogło i przyspieszyło implementację,
	\item GitHub jest wygodnym narzędziem do pracy grupowej i kontrolowania wersji projektu,
	\item Trello pomaga wytyczać cele i przydzielać zadania.
	\end{itemize}
	

	\section{Literatura}
	
\textbf{Asensio MI, Ferragut L., Simon J.:} Modelling of convective phenomena in forest fire. Rev Real Academia de Ciencias, 2002, 96:299–313\\


\end{document}
